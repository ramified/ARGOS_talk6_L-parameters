
\documentclass[reqno,11pt]{amsart}

%\usepackage{color,graphicx}
%\usepackage{mathrsfs,amsbsy}
\usepackage{amssymb}
\usepackage{amsmath}
\usepackage{amsfonts}
\usepackage{bm}
\usepackage{graphicx}
\usepackage{amsthm}
\usepackage{enumerate}
\usepackage[mathscr]{eucal}
\usepackage{float}
\usepackage{mathrsfs}
\usepackage{multicol}
\usepackage{multirow}
\usepackage[all,pdf]{xy}
\usepackage[a4paper,left=3cm,right=3cm]{geometry}
\usepackage[table,xcdraw]{xcolor} % before tikz-cd
\usepackage{tikz-cd}
\usepackage{tikz}
\usetikzlibrary{cd,decorations.pathreplacing}
\usepackage{hyperref}
\usepackage{scalerel}
\usepackage{stackengine,wasysym}
%\usepackage[notcite,notref]{showkeys}

% showkeys  make label explicit on the paper

\makeatletter
\@namedef{subjclassname@2010}{%
  \textup{2010} Mathematics Subject Classification}
\makeatother

\numberwithin{equation}{section}

\theoremstyle{plain}
\newtheorem{theorem}{Theorem}[section]
\newtheorem{lemma}[theorem]{Lemma}
\newtheorem{proposition}[theorem]{Proposition}
\newtheorem{corollary}[theorem]{Corollary}
\newtheorem{claim}[theorem]{Claim}
\newtheorem{defn}[theorem]{Definition}
\newtheorem{ques}[theorem]{Question}
\newtheorem*{bbox}{Black box}
\newtheorem{eg}[theorem]{Example}
\newtheorem*{notation}{Conventions and Notations}

\theoremstyle{plain}
\newtheorem{thmsub}{Theorem}[subsection]
\newtheorem{lemmasub}[thmsub]{Lemma}
\newtheorem{corollarysub}[thmsub]{Corollary}
\newtheorem{propositionsub}[thmsub]{Proposition}
\newtheorem{defnsub}[thmsub]{Definition}

\numberwithin{equation}{section}


\theoremstyle{remark}

\newtheorem{remark}[theorem]{Remark}
\newtheorem{remarks}{Remarks}

%\renewcommand\thefootnote{\fnsymbol{footnote}}
%dont use number as footnote symbol, use this command to change

\DeclareMathOperator{\supp}{supp}
\DeclareMathOperator{\dist}{dist}
\DeclareMathOperator{\vol}{vol}
\DeclareMathOperator{\diag}{diag}
\DeclareMathOperator{\tr}{tr}
\DeclareMathOperator{\Img}{\operatorname{Im}}
\DeclareMathOperator{\Id}{\operatorname{Id}}
\DeclareMathOperator{\Rep}{\operatorname{Rep}}
\DeclareMathOperator{\Mod}{\operatorname{Mod}}
\DeclareMathOperator{\Hom}{\operatorname{Hom}}
\DeclareMathOperator{\Ext}{\operatorname{Ext}}
\DeclareMathOperator{\gldim}{\operatorname{gl.dim}}
\DeclareMathOperator{\projdim}{\operatorname{proj.dim}}
\DeclareMathOperator{\injdim}{\operatorname{inj.dim}}
\DeclareMathOperator{\dimv}{\operatorname{\underline{\mathbf{dim}}}}
\DeclareMathOperator{\Pic}{\operatorname{Pic}}
\DeclareMathOperator{\Jac}{\operatorname{Jac}}
\newcommand{\Spec}{\operatorname{Spec}}
\newcommand{\Irr}{\operatorname{Irr}}
\newcommand{\Gal}{\operatorname{Gal}}
\newcommand{\sep}{\operatorname{sep}}
\newcommand{\norm}[1]{\Vert{#1}\Vert}
\newcommand{\ord}{\operatorname{ord}}
\newcommand{\orde}{\operatorname{ord}_e }
\newcommand{\zdivp}{\mathbb{Z}\!\left[\hspace{-0.5mm}\frac{1}{p}\hspace{-0.5mm}\right]}

\setlength\intextsep{0cm}
\setlength\textfloatsep{0cm}


% from https://tex.stackexchange.com/questions/63545/big-tilde-in-math-mode

\begin{document}
\date{}

\title
{Talk 6: L-parameters
}


\author{Xiaoxiang Zhou}
\address{School of Mathematical Sciences\\
University of Bonn\\
Bonn, 53115\\ Germany\\} 
\email{email:xx352229@mail.ustc.edu.cn}



\setcounter{tocdepth}{1}
\maketitle
\tableofcontents
%%%%%%%%%%%%%%%%%%%%%%%%%%%%%%%%%%%%%%%%%%%%%%%%%%%%%%%%%%%%%%%%%%%%%%%%%%%%%%%%%%%%%%%%%%%%%

First of all, I should apologize that I still do not prepare so well for this talk(so I also \LaTeX$\,$ them after the talk). 

This note records contents in the talk. Feel free to ask me questions and give me typos and suggestions!


\section{Prelude}
During the last five talks, we've discussed a lot about the representation theory of a reductive group $G$ over an non-Archimedean local field $F$, which is viewed as the one side of the local Langlands correspondence (LLC). Today we will focus on the other side of the local Langlands correspondence: Langlands paremeters ($L$-parameters). Roughly speaking, it encodes $1$-cocycles of the Weil group $W_F$ over the dual group $\hat{G}$.
$$\Irr_{\Lambda}\!\big(G(F)\big) \longrightarrow Z^1\big(W_F,\hat{G}(\Lambda)\big)$$

We will not focus on the correspondence in this talk. Instead, we only care about $L$-parameters themselves. After introducing the concepts of Weil group, dual group and $1$-cocycles, we will define a functor of $L$-parameters and show it's represented by a scheme with nice properties. After the break, we will also see the properties of GIT\footnote{GIT=geometric invariant theory.} quotients of $L$-parameters, or, to be precise, we will consider the points and functions of the GIT quotients.

\begin{notation}
Throughout this talk, $F$ is a non-Archimedean local field with residue field $\kappa=\mathbb{F}_q$, $q=p^k$, $G/F$ is a reductive group. $l$ is any prime not equal to $p$, $\Lambda$ is a $\mathbb{Z}_l$-algebra, and $F_n$ denotes the free group generated by $n$ elements.
\end{notation}
\section{Weil group}
The Weil group is defined as a special subgroup of the absolute Galois group $G_F:=\Gal(F^{sep}/F)$, whose structure is already carefully studied and understood well (see \cite{Alex????galois}). Information relevant to this talk is summarized below:
% https://q.uiver.app/?q=WzAsNCxbMCwwLCJGXntzZXB9Il0sWzAsMSwiRl57dHJ9Il0sWzAsMiwiRl57dW59Il0sWzAsMywiRiJdLFsyLDMsIlxcaGF0e1xcbWF0aGJie1p9fSIsMCx7InN0eWxlIjp7ImhlYWQiOnsibmFtZSI6Im5vbmUifX19XSxbMSwyLCJcXGhhdHtcXG1hdGhiYntafX1eeyhwKX0iLDAseyJzdHlsZSI6eyJoZWFkIjp7Im5hbWUiOiJub25lIn19fV0sWzAsMSwiUF9GIiwwLHsic3R5bGUiOnsiaGVhZCI6eyJuYW1lIjoibm9uZSJ9fX1dXQ==
\[\begin{tikzcd}
	{F^{sep}}\arrow[dd, start anchor=mid, end anchor=mid, no head, xshift=-1.6em, decorate, decoration={brace,mirror, amplitude=5, aspect=0.5}, "I_F" left=6pt] \\
	{F^{tr}}\arrow[dd, start anchor=mid, end anchor=mid, no head, xshift=1.8em, decorate, decoration={brace,amplitude=5, aspect=0.5}, "\hat{\mathbb{Z}}^{(p)} \rtimes {\hat{\mathbb{Z}}}" right=6pt] \\
	{F^{un}} \\
	F
	\arrow["{\hat{\mathbb{Z}}}", no head, from=3-1, to=4-1]
	\arrow["{\hat{\mathbb{Z}}^{(p)}}", no head, from=2-1, to=3-1]
	\arrow["{P_F}", no head, from=1-1, to=2-1]
\end{tikzcd}\]
where
\begin{equation*}
\begin{aligned}
  &F^{un}&& \text{is the maximal unramified extension,}  \\ 
  &F^{tr}&&\text{is the maximal tame ramified extension,}  \\ 
  &I_F:=\Gal(F^{sep}/F^{un}) &&\text{is the inertia group,} \\ 
  &P_F:=\Gal(F^{sep}/F^{tr}) &&\text{is the sylow $p$-subgroup of $I_F$, called the wild inertia group,}  \\    
\end{aligned}
\end{equation*}
$$\hat{\mathbb{Z}}= \varprojlim_{n}\mathbb{Z}/n\mathbb{Z}=\prod_l \mathbb{Z}_l, \qquad\qquad \hat{\mathbb{Z}}^{(p)}= \varprojlim_{(n,p)=1}\mathbb{Z}/n\mathbb{Z}=\prod_{l \neq p} \mathbb{Z}_l.$$

We take the geometric Frobenius $\sigma \in \Gal(F^{un}/F)$ as well as a choice of the geometrical generator $\tau \in \Gal(F^{tr}/F^{un})$, and then lift them to the absolute Galois group $G_F$.\footnote{We fix this lift during the whole talk. Also, see [\href{https://math.stackexchange.com/questions/3632345/frobenius-conjugacy-over-inertia-group-modulo-wild-inertia/4442383\#4442383}{stackexchange}] for a proof of the equation $\sigma\tau\sigma^{-1}=\tau^{\frac{1}{q}}$.} We get
$$\Gal (F^{tr}/F) \cong  \hat{\mathbb{Z}}^{(p)} \rtimes {\hat{\mathbb{Z}}} \qquad \sigma\tau\sigma^{-1}=\tau^{\frac{1}{q}}.$$

Until now we have not talked about the Weil group, and any group we mentioned in the tower of fields is given by the usual Krull topology. The Weil group $W_F$ and the related smaller group $W_F^0$ \footnote{I would like to call it the $0$-Weil group, or the skeloton Weil group.} can be viewed as the discretization of the absolute Galois group $G_F$:
\[\begin{tikzcd}[%
    row sep = -1mm
    ,/tikz/column 1/.append style={anchor=base west}
    ]
	{G_F} \\
	\;\rotatebox{90}{$\subseteq$}\\
	{W_F:=\left< I_F,\sigma \right> = \bigsqcup_{g \in \left<\sigma\right>} gI_F} \\
	\;\rotatebox{90}{$\subseteq$}\\	
	{W_F^0:=\left< P_F,\tau, \sigma \right> = \bigsqcup_{g \in \left<\tau,\sigma\right>} gP_F}
\end{tikzcd}\]
To be exact, the topology of $W_F$ is defined such that $I_F \subseteq W_F$ is open and closed, and has the same subspace topology as $I_F \subseteq G_F$; similarly for $W_F^0$. The word "discretization" can be seen clearer once we quotient $P_F$:
\[\begin{tikzcd}[%
    row sep = -1mm
    ,/tikz/column 1/.append style={anchor=base west}
    ]
	{G_F/P_F=\hat{\mathbb{Z}}^{(p)} \rtimes {\hat{\mathbb{Z}}}} \\
	\hspace{5mm}\rotatebox{90}{$\subseteq$}\\
	{W_F/P_F=\hat{\mathbb{Z}}^{(p)} \rtimes \mathbb{Z}} \\
	\hspace{5mm}\rotatebox{90}{$\subseteq$}\\[-1mm]	
	{W_F^0/P_F=\zdivp \rtimes \mathbb{Z}}
\end{tikzcd}\]
\section{(Langlands) Dual group}
\section{1-cocycle}
\section{$\mathcal{Z}^1(W_F,\hat{G}(\Lambda))$ is a good scheme}
\section{GIT and 1-parameter groups}
\section{Geometrical points of GIT quotient}
\section{functions of GIT quotient}

\bibliographystyle{plain}
\bibliography{reference}
\end{document}




