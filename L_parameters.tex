
\documentclass[reqno,11pt]{amsart}

%\usepackage{color,graphicx}
%\usepackage{mathrsfs,amsbsy}
\usepackage{amssymb}
\usepackage{amsmath}
\usepackage{amsfonts}
\usepackage{bm}
\usepackage{graphicx}
\usepackage{amsthm}
\usepackage{enumerate}
\usepackage[mathscr]{eucal}
\usepackage{float}
\usepackage{mathrsfs}
\usepackage{multicol}
\usepackage{multirow}
\usepackage[all,pdf]{xy}
\usepackage[a4paper,left=3cm,right=3cm]{geometry}
\usepackage[table,xcdraw]{xcolor} % before tikz-cd
\usepackage{tikz-cd}
\usepackage{tikz}
\usetikzlibrary{cd,decorations.pathreplacing}
\usepackage{hyperref}
\usepackage{scalerel}
\usepackage{stackengine,wasysym}
%\usepackage[notcite,notref]{showkeys}

% showkeys  make label explicit on the paper
\usepackage{quiver}
\usepackage{MnSymbol}
\usepackage{extpfeil}

\makeatletter
\@namedef{subjclassname@2010}{%
  \textup{2010} Mathematics Subject Classification}
\makeatother

\numberwithin{equation}{section}

\theoremstyle{plain}
\newtheorem{theorem}{Theorem}[section]
\newtheorem{lemma}[theorem]{Lemma}
\newtheorem{proposition}[theorem]{Proposition}
\newtheorem{corollary}[theorem]{Corollary}
\newtheorem{claim}[theorem]{Claim}
\newtheorem{defn}[theorem]{Definition}
\newtheorem{ques}[theorem]{Question}
\newtheorem*{bbox}{Black box}
\newtheorem{eg}[theorem]{Example}
\newtheorem*{notation}{Conventions and Notations}

\theoremstyle{plain}
\newtheorem{thmsub}{Theorem}[subsection]
\newtheorem{lemmasub}[thmsub]{Lemma}
\newtheorem{corollarysub}[thmsub]{Corollary}
\newtheorem{propositionsub}[thmsub]{Proposition}
\newtheorem{defnsub}[thmsub]{Definition}

\numberwithin{equation}{section}


\theoremstyle{remark}

\newtheorem{remark}[theorem]{Remark}
\newtheorem{remarks}{Remarks}

%\renewcommand\thefootnote{\fnsymbol{footnote}}
%dont use number as footnote symbol, use this command to change

\DeclareMathOperator{\supp}{supp}
\DeclareMathOperator{\dist}{dist}
\DeclareMathOperator{\vol}{vol}
\DeclareMathOperator{\diag}{diag}
\DeclareMathOperator{\tr}{tr}
\DeclareMathOperator{\Img}{\operatorname{Im}}
\DeclareMathOperator{\Id}{\operatorname{Id}}
\DeclareMathOperator{\Rep}{\operatorname{Rep}}
\DeclareMathOperator{\Mod}{\operatorname{Mod}}
\DeclareMathOperator{\Hom}{\operatorname{Hom}}
\DeclareMathOperator{\Ext}{\operatorname{Ext}}
\DeclareMathOperator{\gldim}{\operatorname{gl.dim}}
\DeclareMathOperator{\projdim}{\operatorname{proj.dim}}
\DeclareMathOperator{\injdim}{\operatorname{inj.dim}}
\DeclareMathOperator{\dimv}{\operatorname{\underline{\mathbf{dim}}}}
\DeclareMathOperator{\Pic}{\operatorname{Pic}}
\DeclareMathOperator{\Jac}{\operatorname{Jac}}
\newcommand{\Spec}{\operatorname{Spec}}
\newcommand{\Irr}{\operatorname{Irr}}
\newcommand{\Gal}{\operatorname{Gal}}
\newcommand{\Aut}{\operatorname{Aut}}
\newcommand{\Alg}{\operatorname{Alg}}
\newcommand{\Set}{\operatorname{Set}}
\newcommand{\sep}{\operatorname{sep}}
\newcommand{\norm}[1]{\Vert{#1}\Vert}
\newcommand{\ord}{\operatorname{ord}}
\newcommand{\orde}{\operatorname{ord}_e }
\newcommand{\zdivp}{\mathbb{Z}\!\left[\hspace{-0.5mm}\frac{1}{p}\hspace{-0.5mm}\right]}
\newcommand{\GL}{\operatorname{GL}}
\newcommand{\Ugp}{\operatorname{U}}
\setlength\intextsep{0cm}
\setlength\textfloatsep{0cm}


% from https://tex.stackexchange.com/questions/63545/big-tilde-in-math-mode

\begin{document}
\date{}

\title
{Talk 6: L-parameters
}


\author{Xiaoxiang Zhou}
\address{School of Mathematical Sciences\\
University of Bonn\\
Bonn, 53115\\ Germany\\} 
\email{email:xx352229@mail.ustc.edu.cn}



\setcounter{tocdepth}{1}
\maketitle
\tableofcontents
%%%%%%%%%%%%%%%%%%%%%%%%%%%%%%%%%%%%%%%%%%%%%%%%%%%%%%%%%%%%%%%%%%%%%%%%%%%%%%%%%%%%%%%%%%%%%

First of all, I should apologize that I still do not prepare so well for this talk(so I also \LaTeX$\,$ them after the talk). 

This note records contents in the talk. Feel free to ask me questions and give me typos and suggestions!


\section{Prelude}
During the last five talks, we've discussed a lot about the representation theory of a reductive group $G$ over an non-Archimedean local field $F$, which is viewed as the one side of the local Langlands correspondence (LLC). Today we will focus on the other side of the local Langlands correspondence: Langlands paremeters ($L$-parameters). Roughly speaking, it encodes $1$-cocycles of the Weil group $W_F$ over the dual group $\hat{G}$\footnote{Usually people denote $\hat{G}$ for the characters of $G$. Luckily, we won't discuss anything about the representation theory of $G$ in this note, so there won't be any confusion. (Replace $\hat{G}$ by $G^{\vee}$ if you don't like it.)}.
$$\Irr_{\Lambda}\!\big(G(F)\big) \longrightarrow Z^1\big(W_F,\hat{G}(\Lambda)\big)/_\sim$$

We will not focus on the correspondence in this talk. Instead, we only care about $L$-parameters themselves. After introducing the concepts of Weil group, dual group and $1$-cocycles, we will define a functor of $L$-parameters and show it's represented by a scheme with nice properties. After the break, we will also see the properties of GIT\footnote{GIT=geometric invariant theory.} quotients of $L$-parameters, or, to be precise, we will consider the points and functions of the GIT quotients.

\begin{notation}
Throughout this talk, $F$ is a non-Archimedean local field with residue field $\kappa=\mathbb{F}_q$, $q=p^k$, $G/F$ is a reductive group. $l$ is any prime not equal to $p$, $\Lambda$ is a $\mathbb{Z}_l$-algebra, and $F_n$ denotes the free group generated by $n$ elements.
\end{notation}
\section{Weil group}
The Weil group is defined as a special subgroup of the absolute Galois group $G_F:=\Gal(F^{sep}/F)$, whose structure is already carefully studied and understood well (see \cite{Alex????galois}). Information relevant to this talk is summarized below:
% https://q.uiver.app/?q=WzAsNCxbMCwwLCJGXntzZXB9Il0sWzAsMSwiRl57dHJ9Il0sWzAsMiwiRl57dW59Il0sWzAsMywiRiJdLFsyLDMsIlxcaGF0e1xcbWF0aGJie1p9fSIsMCx7InN0eWxlIjp7ImhlYWQiOnsibmFtZSI6Im5vbmUifX19XSxbMSwyLCJcXGhhdHtcXG1hdGhiYntafX1eeyhwKX0iLDAseyJzdHlsZSI6eyJoZWFkIjp7Im5hbWUiOiJub25lIn19fV0sWzAsMSwiUF9GIiwwLHsic3R5bGUiOnsiaGVhZCI6eyJuYW1lIjoibm9uZSJ9fX1dXQ==
\[\begin{tikzcd}
	{F^{sep}}\arrow[dd, start anchor=mid, end anchor=mid, no head, xshift=-1.6em, decorate, decoration={brace,mirror, amplitude=5, aspect=0.5}, "I_F" left=6pt] \\
	{F^{tr}}\arrow[dd, start anchor=mid, end anchor=mid, no head, xshift=1.8em, decorate, decoration={brace,amplitude=5, aspect=0.5}, "\hat{\mathbb{Z}}^{(p)} \rtimes {\hat{\mathbb{Z}}}" right=6pt] \\
	{F^{un}} \\
	F
	\arrow["{\hat{\mathbb{Z}}}", no head, from=3-1, to=4-1]
	\arrow["{\hat{\mathbb{Z}}^{(p)}}", no head, from=2-1, to=3-1]
	\arrow["{P_F}", no head, from=1-1, to=2-1]
\end{tikzcd}\]
where
\begin{equation*}
\begin{aligned}
  &F^{un}&& \text{is the maximal unramified extension,}  \\ 
  &F^{tr}&&\text{is the maximal tame ramified extension,}  \\ 
  &I_F:=\Gal(F^{sep}/F^{un}) &&\text{is the inertia group,} \\ 
  &P_F:=\Gal(F^{sep}/F^{tr}) &&\text{is the sylow $p$-subgroup of $I_F$, called the wild inertia group,}  \\    
\end{aligned}
\end{equation*}
$$\hat{\mathbb{Z}}= \varprojlim_{n}\mathbb{Z}/n\mathbb{Z}=\prod_l \mathbb{Z}_l, \qquad\qquad \hat{\mathbb{Z}}^{(p)}= \varprojlim_{(n,p)=1}\mathbb{Z}/n\mathbb{Z}=\prod_{l \neq p} \mathbb{Z}_l.$$

We take the geometric Frobenius $\sigma \in \Gal(F^{un}/F)$ as well as a choice of the geometrical generator $\tau \in \Gal(F^{tr}/F^{un})$, and then lift them to the absolute Galois group $G_F$.\footnote{We fix this lift during the whole talk. Also, see [\href{https://math.stackexchange.com/questions/3632345/frobenius-conjugacy-over-inertia-group-modulo-wild-inertia/4442383\#4442383}{stackexchange}] for a proof of the equation $\sigma\tau\sigma^{-1}=\tau^{\frac{1}{q}}$.} We get
$$\Gal (F^{tr}/F) \cong  \hat{\mathbb{Z}}^{(p)} \rtimes {\hat{\mathbb{Z}}} \qquad \sigma\tau\sigma^{-1}=\tau^{\frac{1}{q}}.$$

Until now we have not talked about the Weil group, and any group we mentioned in the tower of fields is given by the usual Krull topology. The Weil group $W_F$ and the related smaller group $W_F^0$ \footnote{I would like to call it the $0$-Weil group, or the skeloton Weil group.} can be viewed as the discretization of the absolute Galois group $G_F$:
\[\begin{tikzcd}[%
    row sep = -1mm
    ,/tikz/column 1/.append style={anchor=base west}
    ]
	{G_F} \\
	\;\rotatebox{90}{$\subseteq$}\\
	{W_F:=\left< I_F,\sigma \right> = \bigsqcup_{g \in \left<\sigma\right>} gI_F} \\
	\;\rotatebox{90}{$\subseteq$}\\	
	{W_F^0:=\left< P_F,\tau, \sigma \right> = \bigsqcup_{g \in \left<\tau,\sigma\right>} gP_F}
\end{tikzcd}\]
To be exact, the topology of $W_F$ is defined such that $I_F \subseteq W_F$ is open and closed, and has the same subspace topology as $I_F \subseteq G_F$; similarly for $W_F^0$. The word "discretization" can be seen clearer once we quotient $P_F$:
\[\begin{tikzcd}[%
    row sep = -1mm
    ,/tikz/column 1/.append style={anchor=base west}
    ]
	{G_F/P_F=\hat{\mathbb{Z}}^{(p)} \rtimes {\hat{\mathbb{Z}}}} \\
	\hspace{5mm}\rotatebox{90}{$\subseteq$}\\
	{W_F/P_F=\hat{\mathbb{Z}}^{(p)} \rtimes \mathbb{Z}} \\
	\hspace{5mm}\rotatebox{90}{$\subseteq$}\\[-1mm]	
	{W_F^0/P_F=\zdivp \rtimes \mathbb{Z}}
\end{tikzcd}\]
\section{(Langlands) Dual group}
The second ingredient of L-parameters is the dual group. The main reference are \cite[Section 4-5]{casselman2001group} and \cite{kevin2012unitary}. If you have never seen the structure theory of reductive groups, this \href{http://www.math.columbia.edu/~makisumi/old/reductivegroups.pdf}{expository paper} is highly recommanded.

\begin{defn}[Dual group, Langlands dual group]
Fix a quasi-split reductive connected group\footnote{Later, when (quasi-)split group is mentioned, it's always assumed to be a reductive connected group.} $G$ over $F$, a Borel subgroup $B \leq G$ as well as a maximal torus $T \leq B$. We get a root system
$$\left( X^*(T), \Delta(B), X_*(T), \Delta^{\vee}(B) \right)$$ 
with a $G_F$-action. Temporarily forget the $G_F$-action, there is a unique split group $\hat{G}$ (with standard Borel $\hat{B}$ and standard tori $\hat{T}$) whose root system
$$\left( X^*(\hat{T}), \Delta(\hat{B}), X_*(\hat{T}), \Delta^{\vee}(\hat{B}) \right) \cong \left( X_*(T), \Delta^{\vee}(B), X^*(T), \Delta(B) \right)$$ 
dual to the root system of $G$. $\hat{G}$ is called the \textbf{dual group} of $G$. We attach $\hat{G}$ with a $G_F$-action such that the induced $G_F$-action on the root system is the same as the original one. Finally, the \textbf{Langlands dual group} ${^L}G$ is defined as the semidirect product of $\hat{G}$ and $W_F$, where $W_F$ acts on $\hat{G}$ as a subgroup of $G_F$. 
\end{defn}
The whole process can be encapsulated in the following diagram:
%https://tex.stackexchange.com/questions/289295/column-alignment-in-tikzcd Useful!
% https://q.uiver.app/?q=WzAsOCxbMSwwLCJcXGxlZnQoIFheKihUKSwgXFxEZWx0YShCKSwgWF8qKFQpLCBcXERlbHRhXntcXHZlZX0oQikgXFxyaWdodClcXHRleHR7IFxccm90YXRlYm94W29yaWdpbj1jXXs5MH17JFxcY2lyY2xlYXJyb3dyaWdodCR9fVxcOyBHX0YiXSxbMSwxLCJcXGxlZnQoWF8qKFQpLCBcXERlbHRhXntcXHZlZX0oQiksIFheKihUKSwgXFxEZWx0YShCKSBcXHJpZ2h0KVxcdGV4dHsgXFxyb3RhdGVib3hbb3JpZ2luPWNdezkwfXskXFxjaXJjbGVhcnJvd3JpZ2h0JH19XFw7IEdfRiJdLFsxLDIsIlxcaGF0e0d9XFx0ZXh0eyBcXHJvdGF0ZWJveFtvcmlnaW49Y117OTB9eyRcXGNpcmNsZWFycm93cmlnaHQkfX1cXDsgR19GIFxcbG9uZ2xlZnRhcnJvdyBXX0YiXSxbMSwzLCJ7Xkx9Rzo9IFxcaGF0e0d9IFxccnRpbWVzIFdfRiJdLFswLDNdLFswLDJdLFswLDFdLFswLDAsIkcvRiJdLFs0LDMsIiIsMCx7InN0eWxlIjp7ImJvZHkiOnsibmFtZSI6InNxdWlnZ2x5In19fV0sWzUsMiwiXFx0ZXh0e2ZvcmdldGF0dGFjaH0iLDIseyJzdHlsZSI6eyJib2R5Ijp7Im5hbWUiOiJzcXVpZ2dseSJ9fX1dLFs2LDEsIlxcdGV4dHtkdWFsfSIsMCx7InN0eWxlIjp7ImJvZHkiOnsibmFtZSI6InNxdWlnZ2x5In19fV0sWzcsMCwiKFQsQikiLDAseyJzdHlsZSI6eyJib2R5Ijp7Im5hbWUiOiJzcXVpZ2dseSJ9fX1dXQ==
\[\begin{tikzcd}[column sep=large,row sep=0mm,/tikz/column 1/.append style={anchor=base east},/tikz/column 2/.append style={anchor=base west}]
	{G/F} & {\left( X^*(T), \Delta(B), X_*(T), \Delta^{\vee}(B) \right)\text{ \rotatebox[origin=c]{90}{$\circlearrowright$}}\; G_F} \\
	{} & {\left(X_*(T), \Delta^{\vee}(B), X^*(T), \Delta(B) \right)\text{ \rotatebox[origin=c]{90}{$\circlearrowright$}}\; G_F} \\
	{} & {\hat{G}\text{ \rotatebox[origin=c]{90}{$\circlearrowright$}}\; G_F \longleftarrow W_F} \\
	{} & {{^L}G:= \hat{G} \rtimes W_F}
	\arrow[squiggly, from=4-1, to=4-2]
	\arrow["{\text{forget}}","{\text{attach}}"', squiggly, from=3-1, to=3-2]
	\arrow["{\text{dual}}", squiggly, from=2-1, to=2-2]
	\arrow["{(T,B)}", squiggly, from=1-1, to=1-2]
\end{tikzcd}\]

\begin{eg}

% https://q.uiver.app/?q=WzAsOSxbMCwwLCJHL0YiXSxbMSwwLCJcXGhhdHtHfS9cXG1hdGhiYntafV9sIl0sWzIsMCwie15MfUcvXFxtYXRoYmJ7Wn1fbCJdLFswLDEsIlxcR0xfbiJdLFsxLDEsIlxcR0xfbiJdLFsxLDIsIlxcR0xfbiJdLFsyLDEsIlxcR0xfbiBcXHRpbWVzIFdfRiJdLFsyLDIsIlxcR0xfbiBcXHJ0aW1lcyBXX0YiXSxbMCwyLCJcXFVncChuLEUvRikiXSxbMCwxLCIiLDAseyJzdHlsZSI6eyJib2R5Ijp7Im5hbWUiOiJzcXVpZ2dseSJ9fX1dLFszLDQsIiIsMCx7InN0eWxlIjp7ImJvZHkiOnsibmFtZSI6InNxdWlnZ2x5In19fV0sWzEsMiwiIiwwLHsic3R5bGUiOnsiYm9keSI6eyJuYW1lIjoic3F1aWdnbHkifX19XSxbNCw2LCIiLDAseyJzdHlsZSI6eyJib2R5Ijp7Im5hbWUiOiJzcXVpZ2dseSJ9fX1dLFs1LDcsIiIsMCx7InN0eWxlIjp7ImJvZHkiOnsibmFtZSI6InNxdWlnZ2x5In19fV0sWzgsNSwiIiwwLHsic3R5bGUiOnsiYm9keSI6eyJuYW1lIjoic3F1aWdnbHkifX19XV0=
\[\begin{tikzcd}[column sep=3em,row sep=0cm]
	{G/F} & {\hat{G}/\mathbb{Z}_l} & {{^L}G/\mathbb{Z}_l} \\
	{\GL_n} & {\GL_n} & {\GL_n \times W_F} \\
	{\Ugp(n,E/F)} & {\GL_n} & {\GL_n \rtimes W_F}
	\arrow[squiggly, from=1-1, to=1-2]
	\arrow[squiggly, from=2-1, to=2-2]
	\arrow[squiggly, from=1-2, to=1-3]
	\arrow[squiggly, from=2-2, to=2-3]
	\arrow[squiggly, from=3-2, to=3-3]
	\arrow[squiggly, from=3-1, to=3-2]
\end{tikzcd}\]

Here, $E/F$ is a degree $2$ Galois extension (of NA local field), $\Ugp(n,E/F)$ is an quasi-split group defined by
$$\Ugp(n,E/F)(R)=\left\{ A=(a_{ij})_{i,j=1}^n  \;\middle|\;
\begin{aligned}
&\\[-5mm]
& a_{ij} \in E \otimes_F R\; \\[-1mm]
& A\omega A^H =\omega
\end{aligned}
 \right\}$$
 where
 $$\omega:= \left[
 \begin{smallmatrix}
  &&&1 \\ &&-1& \\&1&&\\ \udots &&&
  \end{smallmatrix}\right] \qquad \Gal(E/F):=\{1,\gamma\} \qquad A^H :=\gamma (A^T).$$
  In this case, the Weil group $W_F$ acts on $\GL_n$ by
  $$W_F \xtwoheadrightarrow{\hspace{0.2cm}} \Gal(E/F)=\{1,\gamma\} \; \rotatebox[origin=c]{270}{$\circlearrowleft$} \;\GL_n \qquad \gamma: A \longmapsto (\omega^{-1}A^{-1}\omega)^T$$
  The details can be found in \cite{kevin2012unitary,casselman2001group}, or \href{https://github.com/ramified/personal_handwritten_collection/blob/main/weeklyupdate/2022.05.29_unitary_group.pdf}{my calculation for $\Ugp(3,E/F)$}.
\end{eg}

\begin{remark}
Quasi-split group always becomes split after some finite extension. Therefore, the action of the Weil group $W_F$ always factors through a finite quotient $Q$. In many articles the Langlands dual group is defined as $\hat{G} \rtimes Q$, to obtain the structure of finite type scheme.\footnote{The main problem is, we have topology on both $\hat{G}$ and $W_F$, and the ``correct" semidirect product should be compatible with those topologies. It's also not perfect ot consider them as abstract groups, since $\hat{G}$ is a scheme and there's no nature group structure even on the closed points of $\hat{G}$.}%See https://math.stackexchange.com/questions/2396430/multiplication-of-two-closed-points-of-a-group-scheme
\end{remark}

As you see, the Langlands dual group encodes information of $W_F$-action, and the semidirect product inside twists everything to be less understandable. But in this talk, the real difficulty happens even when you consider $G=\GL_n$ case, so feel free to make your life easier, and replace $\rtimes$ by $\times$ whenever you get troubles.


\section{1-cocycle}
The third ingredient is easier to define.\footnote{We omit everything about Galois cohomology in \cite{serre1979galois}. Here we list two formulas for comparison:
$$H^0(W,A)=A^W, \qquad H^0(W,A)=Z^1(W,A)/A.$$} Let us begin in a little more general.

\begin{defn}
Let $W$, $A$ be topological groups, and let $W$ acts on $A$ by 
$$\phi: W \longrightarrow \Aut(A) \qquad \gamma \longrightarrow {^\gamma}(-).$$
The 1-cocycle is defined as
\begin{equation*}
\begin{aligned}
  Z^1(W,A):=\;&  \left\{ 
  \begin{aligned}
  &\\[-5mm]
   {^L}\varphi:W& \longrightarrow A\rtimes W \\[-1mm]
  \gamma & \longmapsto (\gamma_0,\gamma):={^L}\gamma
  \end{aligned}
   \;\middle|\; 
   \begin{aligned}
     &\\[-5mm]
     &{^L}\varphi:\text{continuous group homo} \\[-1mm]
     &{^L}\varphi \text{ is a section}
     \end{aligned}
       \;\right\} \\
  =\;&  \left\{ \varphi:W \longrightarrow A \;\middle|\; \varphi(\gamma\gamma')=\varphi(\gamma)\,{^\gamma\hspace{-1mm}}\left(\varphi(\gamma)\right)  \;\right\} \\
\end{aligned}
\end{equation*}
\end{defn} 
Finally we can define L-parameters.\footnote{Strictly speaking, we replace topological space by condensed things in the following definition. In Section \ref{sec:GITquotient}, the same definition applies for $\hat{P}$, $\hat{B}$, or $\hat{T}$.}
\begin{defn}
Denote $W$ as any subquotient group of $W_F$ which can (naturally) act on $\hat{G}$, we define a functor 
$$\mathcal{Z}^1(W,\hat{G}): \mathbb{Z}_l\text{-}\Alg \longrightarrow \Set \qquad \Lambda \longmapsto Z^1\big(W_F,\hat{G}(\Lambda)\big),$$
and elements in $Z^1\big(W_F,\hat{G}(\Lambda)\big)$ are called L-parameters.
\end{defn}
\begin{eg}
The $\Lambda$-points of $\mathcal{Z}^1(W,\GL_n)$ are just $n$-dim (continuous?) representations of $W$ (with coefficient in $\Lambda$), i.e.,
$$\mathcal{Z}^1(W,\GL_n)(\Lambda)=\big\{ \rho:W\longrightarrow \GL_n(\Lambda)  \big\}.$$
\end{eg}
In the next section, we will prove the representability of $\mathcal{Z}^1(W_F,\hat{G})$.
\section{$\mathcal{Z}^1(W_F,\hat{G})$ is a good scheme}
\section{Interlude: GIT and 1-parameter groups}
\section{Geometrical points of GIT quotient}\label{sec:GITquotient}
\section{functions of GIT quotient}

\bibliographystyle{plain}
\bibliography{reference}
\end{document}




